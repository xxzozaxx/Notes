% Created 2019-05-30 Thu 15:06
% Intended LaTeX compiler: xelatex
\documentclass[11pt, a4paper]{article}
\usepackage{graphicx}
\usepackage{grffile}
\usepackage{longtable}
\usepackage{wrapfig}
\usepackage{rotating}
\usepackage[normalem]{ulem}
\usepackage{amsmath}
\usepackage{textcomp}
\usepackage{amssymb}
\usepackage{capt-of}
\usepackage{hyperref}
\usepackage{fontspec}
\setmainfont{EB Garamond}
\usepackage[margin=20mm]{geometry}
\usepackage{amsthm}
\renewcommand\qedsymbol{$\blacksquare$}
\newtheorem{theorem}{Theorem}
\newtheorem{Lemma}[theorem]{Lemma}
\author{Ahmed Khaled}
\date{\today}
\title{Mathematical Foundation}
\hypersetup{
 pdfauthor={Ahmed Khaled},
 pdftitle={Mathematical Foundation},
 pdfkeywords={},
 pdfsubject={},
 pdfcreator={Emacs 27.0.50 (Org mode 9.2.3)}, 
 pdflang={English}}
\begin{document}

\maketitle
\begin{abstract}
This Document contain my notes about Axioms, Definitions and basic theories.
\end{abstract}

\section{Real Numbers}
\label{sec:org71e7ac2}
\subsection{Fields}
\label{sec:org6fb7fdd}
In rigorous mathematics real number is a set of numbers defined as a complete, ordered field

\begin{itemize}
\item \textsc{Def}. \emph{Field} is a non-empty set on which two \(\overline{\mbox{binary operation}}\) are
defined \marginpar{refer to Group theory and Set theory TODO}

\item \textsc{Def}. \emph{Binary Operation} in field \(\mathbb{F}\) is a function that "take"
an ordered pair of element and "return" an element in \(\mathbb{F}\), and it said to be
the operation on the set whose both domain and co-domain in the same set.
\end{itemize}
\[ \forall a,b \in \mathbb{F} (\exists c \in \mathbb{F}) : (c = a \circ b) \]

\begin{itemize}
\item the 9 golden basic most primitive axioms:
\begin{enumerate}
\item \textsc{Axi}. \emph{Associative law for addition} \(( a + b) + c = a + ( a + c )\)
\item \textsc{Axi}. \emph{Existence of additive identity} \(\exists 0:  a + 0 = 0 + a = a\)
\item \textsc{Axi}. \emph{Existence of additive inverse} \(\forall a \in \mathbb{R} \exists (-a) : a + (-a) = (-a) + a = 0\)
\item \textsc{Axi}. \emph{Commutative law of addition} \(a + b = b + a\)
\item \textsc{Axi}. \emph{Associative law for multiplication} \(( a \cdot b) \cdot c = a \cdot ( a \cdot c )\)
\item \textsc{Axi}. \emph{Existence of multiplicative identity} \(\exists 1 \neq 0:  a \cdot 1 = 1 \cdot a = a\)
\item \textsc{Axi}. \emph{Existence of multiplicative inverse} \(\forall a \neq 0 \in \mathbb{R} \exists (a^{-1}) : a + (a^{-1}) = (a^{-1}) + a = 0\)
\item \textsc{Axi}. \emph{Commutative law of multiplication} \(a \cdot b = b \cdot a\)
\item \textsc{Axi}. \emph{Distributive law} \(a \cdot ( b + c ) = a \cdot b + a \cdot c\)
\end{enumerate}

\item Theorem
\end{itemize}

\begin{theorem}
  $ \forall a \in \mathbb{F}: a \cdot 0 = 0 $
\end{theorem}

\begin{proof}
  using axiom Num.9
  \begin{align*}
    a \cdot 0 &= a \cdot (0 + 0) \\
          &= a \cdot 0 + a \cdot 0 \\
  \end{align*}
by adding $-(a \cdot 0)$ to both side
\[ a \cdot 0 = 0 \]
\end{proof}
\end{document}